\documentclass{article}
\usepackage{graphicx}
\usepackage{subcaption}
\usepackage[margin=1in]{geometry}
\title{CS 4650/7650 ECE 4655/7655 Digital Image Processing 2025 Homework 1A: Point Processes in Python}
\author{David Allen Stephan Marino}
\begin{document}
\maketitle
\newpage
\section{Abstract}
The student (David Marino) applies Intensity Transformation processes on Human Lung Carcinoma Cells to segment the details present.
Utilizing thresholding, the provided cell images from Florida State University are segmented to highlight the different parts of the Carcinoma cells.
The original image is the result of individual color channels pseudocolored to correspond to the different fluorophore emission spectral profiles.
The nuclei of the cells are highlighted in red, the mitochondria are highlighted in green, and the membrane is highlighted in blue.
\newpage
\section{Introduction}
The student (David Marino) applies thresholding image processing techniques to extract highlight information from the cell images.
Utilizing binary images and thresholding, several different metrics about the cells can be extracted.
The student begins by converting the original input RGB image to gray for use in thresholding.
Using the output gray image, thresholding is applied to highlight the cell groupings.
Afterwards, the color channels of the original RGB image are extracted, and thresholding is applied once more to each color channel.
Doing this shows the borders and details of each region found in the cell.
Thresholding enables segmentation of the different fluorophore emission spectral profiles for further use in experiments.
\newpage
\section{Experiments and Results}
\begin{figure}[h!]
\centering
\begin{subfigure}{0.3\textwidth}
  \includegraphics[width=\linewidth]{a549large12}
  \caption{Input RGB image.}
\end{subfigure}\hfill
\begin{subfigure}{0.3\textwidth}
  \includegraphics[width=\linewidth]{a549large12_gray}
  \caption{Output grayscale image.}
\end{subfigure}\hfill
\begin{subfigure}{0.3\textwidth}
  \includegraphics[width=\linewidth]{a549large12_thresh_32.jpg}
  \caption{Binary image thresh 32.}
\end{subfigure}
\begin{subfigure}{0.3\textwidth}
  \includegraphics[width=\linewidth]{a549large12_thresh_64.jpg}
  \caption{Binary image thresh 64.}
\end{subfigure}\hfill
\begin{subfigure}{0.3\textwidth}
  \includegraphics[width=\linewidth]{a549large12_thresh_96.jpg}
  \caption{Binary image thresh 96.}
\end{subfigure}\hfill
\begin{subfigure}{0.3\textwidth}
  \includegraphics[width=\linewidth]{a549large12_thresh_128.jpg}
  \caption{Binary image thresh 128.}
\end{subfigure}
\begin{subfigure}{0.3\textwidth}
  \includegraphics[width=\linewidth]{a549large12_thresh_160.jpg}
  \caption{Binary image thresh 160.}
\end{subfigure}\hfill
\begin{subfigure}{0.3\textwidth}
  \includegraphics[width=\linewidth]{a549large12_thresh_196.jpg}
  \caption{Binary image thresh 196.}
\end{subfigure}\hfill
\begin{subfigure}{0.3\textwidth}
  \includegraphics[width=\linewidth]{a549large12_thresh_224.jpg}
  \caption{Binary image thresh 224.}
\end{subfigure}
\caption{Thresholding results for Human Lung Carcinoma Cells (A-549).}
\end{figure}
\newpage
\begin{figure}[h!]
\centering
\begin{subfigure}{0.3\textwidth}
  \includegraphics[width=\linewidth]{red_a549large12_thresh_32.jpg}
  \caption{Red channel thresh 32.}
\end{subfigure}\hfill
\begin{subfigure}{0.3\textwidth}
  \includegraphics[width=\linewidth]{red_a549large12_thresh_64.jpg}
  \caption{Red channel thresh 64.}
\end{subfigure}\hfill
\begin{subfigure}{0.3\textwidth}
  \includegraphics[width=\linewidth]{red_a549large12_thresh_96.jpg}
  \caption{Red channel thresh 96.}
\end{subfigure}
\begin{subfigure}{0.3\textwidth}
  \includegraphics[width=\linewidth]{red_a549large12_thresh_128.jpg}
  \caption{Red channel thresh 128.}
\end{subfigure}\hfill
\begin{subfigure}{0.3\textwidth}
  \includegraphics[width=\linewidth]{red_a549large12_thresh_160.jpg}
  \caption{Red channel thresh 160.}
\end{subfigure}\hfill
\begin{subfigure}{0.3\textwidth}
  \includegraphics[width=\linewidth]{red_a549large12_thresh_196.jpg}
  \caption{Red channel thresh 196.}
\end{subfigure}
\begin{subfigure}{0.3\textwidth}
  \includegraphics[width=\linewidth]{red_a549large12_thresh_224.jpg}
  \caption{Red channel thresh 224.}
\end{subfigure}
\caption{Thresholding results for the red channel of Human Lung Carcinoma Cells (A-549).}
\end{figure}
\newpage
\begin{figure}[h!]
\centering
\begin{subfigure}{0.3\textwidth}
  \includegraphics[width=\linewidth]{green_a549large12_thresh_32.jpg}
  \caption{Green channel thresh 32.}
\end{subfigure}\hfill
\begin{subfigure}{0.3\textwidth}
  \includegraphics[width=\linewidth]{green_a549large12_thresh_64.jpg}
  \caption{Green channel thresh 64.}
\end{subfigure}\hfill
\begin{subfigure}{0.3\textwidth}
  \includegraphics[width=\linewidth]{green_a549large12_thresh_96.jpg}
  \caption{Green channel thresh 96.}
\end{subfigure}
\begin{subfigure}{0.3\textwidth}
  \includegraphics[width=\linewidth]{green_a549large12_thresh_128.jpg}
  \caption{Green channel thresh 128.}
\end{subfigure}\hfill
\begin{subfigure}{0.3\textwidth}
  \includegraphics[width=\linewidth]{green_a549large12_thresh_160.jpg}
  \caption{Green channel thresh 160.}
\end{subfigure}\hfill
\begin{subfigure}{0.3\textwidth}
  \includegraphics[width=\linewidth]{green_a549large12_thresh_196.jpg}
  \caption{Green channel thresh 196.}
\end{subfigure}
\begin{subfigure}{0.3\textwidth}
  \includegraphics[width=\linewidth]{green_a549large12_thresh_224.jpg}
  \caption{Green channel thresh 224.}
\end{subfigure}
\caption{Thresholding results for the green channel of Human Lung Carcinoma Cells (A-549).}
\end{figure}
\newpage
\begin{figure}[h!]
\centering
\begin{subfigure}{0.3\textwidth}
  \includegraphics[width=\linewidth]{blue_a549large12_thresh_32.jpg}
  \caption{Blue channel thresh 32.}
\end{subfigure}\hfill
\begin{subfigure}{0.3\textwidth}
  \includegraphics[width=\linewidth]{blue_a549large12_thresh_64.jpg}
  \caption{Blue channel thresh 64.}
\end{subfigure}\hfill
\begin{subfigure}{0.3\textwidth}
  \includegraphics[width=\linewidth]{blue_a549large12_thresh_96.jpg}
  \caption{Blue channel thresh 96.}
\end{subfigure}
\begin{subfigure}{0.3\textwidth}
  \includegraphics[width=\linewidth]{blue_a549large12_thresh_128.jpg}
  \caption{Blue channel thresh 128.}
\end{subfigure}\hfill
\begin{subfigure}{0.3\textwidth}
  \includegraphics[width=\linewidth]{blue_a549large12_thresh_160.jpg}
  \caption{Blue channel thresh 160.}
\end{subfigure}\hfill
\begin{subfigure}{0.3\textwidth}
  \includegraphics[width=\linewidth]{blue_a549large12_thresh_196.jpg}
  \caption{Blue channel thresh 196.}
\end{subfigure}
\begin{subfigure}{0.3\textwidth}
  \includegraphics[width=\linewidth]{blue_a549large12_thresh_224.jpg}
  \caption{Blue channel thresh 224.}
\end{subfigure}
\caption{Thresholding results for the blue channel of Human Lung Carcinoma Cells (A-549).}
\end{figure}
\newpage
\section{Conclusions}
\subsection{Did the programs work as expected?}
The programs ./rgb2gray and ./gray2binary worked as expected.
The binary images of the different channels show distinct regions.
The red regions, in particular, are clear, and the optimal thresholding falls between 128 and 160.
Before 128, these regions do show some noise, and after 160, the regions start to lose detail.
The blue regions are quite complex because the color is not densely populated but loosely coupled, and the optimal thresholding falls between 0 and 32.
Above 32, and the regions instantly begin losing detail however little noise is present.
The green regions are highly populated and complex, and clear borders are not seen until thresholding is above 64.
Noise is present at 64, and the borders continue to become more defined between 96 and 128.
However, detail is lost between 96 and 128.
Overall, the results turn out good and with further optimization of threshold values the regions can become more distinct.
\subsection{Are the results satisfactory? Why/Why not?}
The results are satisfactory. Using thresholding to segment the different regions of the cells, clear distinctions can be seen.
These results can be further utilized for research and highlight the regions clearly.
The green and blue regions of the cell, however, do not come through as clearly as the red regions.
This is because those regions are noisier and are scattered and intermixed with each other.
The red regions stand out clearly as they were densely populated and regionally contained within themselves.
\newpage
\section{References}
\begin{itemize}
  \item Dr. Filiz Bunyak\\
        \textit{Python for Image Processing and Vision}.
  \item OpenCV documentation\\
        \textit{https://docs.opencv.org/4.x/}.
  \item PyPlot documentation\\
        \textit{https://matplotlib.org/stable/tutorials/pyplot.html}
  \item Python documentation\\
        \textit{https://docs.python.org/3/}
  \item JupyterLab documentation\\
        \textit{https://jupyterlab.readthedocs.io/en/latest/}
  \item Alireza Heidari\\
        \textit{Three–dimensional (3D) imaging spectroscopy of carcinoma, sarcoma, leukemia, lymphoma, multiple myeloma, melanoma, brain and spinal cord tumors, germ cell tumors, neuroendocrine tumors and carcinoid tumors under synchrotron radiation}\\
        California South Universit, January 29, 2019
  \item Michael W. Davidson\\
        \textit{Fluorescence Digital Image Gallery}\\
        Florida State University, October 14, 2004
\end{itemize}
\end{document}
